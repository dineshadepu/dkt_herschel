\documentclass[preprint,12pt]{elsarticle}
% \documentclass[draft,12pt]{elsarticle}

\usepackage{hyperref}
\usepackage{graphicx}
\usepackage{subcaption}
\usepackage{amssymb}
\usepackage{amsmath}
\usepackage{multirow}
\usepackage{relsize}
\usepackage[utf8]{inputenc}
\usepackage{cleveref}
\usepackage{algorithm}
\usepackage[noend]{algpseudocode}
\usepackage[section]{placeins}
\usepackage{booktabs}
\usepackage{url}

% For the TODOs
\usepackage{xcolor}
\usepackage{xargs}
\usepackage[colorinlistoftodos,textsize=footnotesize]{todonotes}
\newcommand{\todoin}{\todo[inline]}
% from here: https://tex.stackexchange.com/questions/9796/how-to-add-todo-notes
\newcommandx{\unsure}[2][1=]{\todo[linecolor=red,backgroundcolor=red!25,bordercolor=red,#1]{#2}}
\newcommandx{\change}[2][1=]{\todo[linecolor=blue,backgroundcolor=blue!25,bordercolor=blue,#1]{#2}}
\newcommandx{\info}[2][1=]{\todo[linecolor=OliveGreen,backgroundcolor=OliveGreen!25,bordercolor=OliveGreen,#1]{#2}}

%Boldtype for greek symbols
\newcommand{\teng}[1]{\ensuremath{\boldsymbol{#1}}}
\newcommand{\ten}[1]{\ensuremath{\mathbf{#1}}}

\usepackage{lineno}
% \linenumbers
\UseRawInputEncoding

\journal{}

\begin{document}

\begin{frontmatter}

  \title{DKT in non-Newtonian fluids with SPH}
  \author[]{}
  \ead{}
  \author[]{}
  \ead{}
\address[UoS]{}

\cortext[cor1]{Corresponding author}


\begin{abstract}

\end{abstract}

\begin{keyword}
%% keywords here, in the form: keyword \sep keyword
{}, {}, {SPH}, {}

%% MSC codes here, in the form: \MSC code \sep code
%% or \MSC[2008] code \sep code (2000 is the default)

\end{keyword}

\end{frontmatter}

% \linenumbers

\section{Introduction}
\label{sec:intro}

The following work in modeling of non-Newtonian fluid with in SPH is carried
out by the following authors


\begin{itemize}
\item \citet{park_semi-implicit_2020} modeled Herschel Bulkley flows with SPH
  using a semi-implicit operator technique.
\item \citet{mao_gpu-accelerated_2022} modeled non-Newtonian flows in SPH for
  GPUs, however it doesn't model the viscous terms with an implicit
  operator technique.
\item \citet{bellaard_simulation_2021} modeled viscosity in SPH using a
  semi-implicit operator technique, however it is not sure if they modeled
  non-Newtonian flows.
\item \citet{han_improved_2013} modeled viscosity in SPH using a semi-implicit
  operator technique.
\item \citet{wu_modelling_2020} couples DEM with SPH and models the movement
  of particles in non-Newtonian flows. This work doesn't have GPU formulation
  and no implicit operator formulation is considered.
\item \citet{hui_drafting_2022} models DKT in non-Newtonian fluids with IB-LBM
  approach.
\item \citep{shi_gpu-based_2022} models non-Newtonian multiphase flows with
  delta-Plus-SPH Model on GPU. However, the viscosity term is not implicit.
\item \citep{jiao_numerical_2022} models polygonal particles settling within
  non-Newtonian fluids with IB-LBM-DEM.
\item \citep{rossi_sph_2022} models flow past cylinder of thixo-viscoplastic
  fluid flow.
\item \citep{dietemann_smoothed_2020} Proposed implicit viscous coupled to
  rigid solver algorithm to model rigid bodies in viscous flows.
\end{itemize}




\FloatBarrier%
\section{Conclusions}
\label{sec:conclusions}


\section*{References}


\bibliographystyle{model6-num-names}
\bibliography{references}
\end{document}

%%% Local Variables:
%%% mode: latex
%%% TeX-master: "paper"
%%% fill-column: 78
%%% End:
